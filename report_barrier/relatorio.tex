%%
%% If you intend to use figures of formats jpg, png or pdf and want the
%% output to be immediately a pdf file, compile with pdflatex.
%%
%% If you want to use eps or ps figures, your output will be a dvi
%% file that can be converted to ps and pdf formats. In this case you
%% should compile your document with latex.
%%
%%This template is compatible with both methods.
%%
\title{{\normalsize Optimization and Algorithms} \\
Financial Portfolio Optimization}
\author{
        Bernardo Gomes
        \and
        Kishan Rama
        \and
        Tom\'as Falcato \\ {\tt bernardo.n.gomes@ist.utl.pt} \and {\tt kishan.rama@ist.utl.pt} \and {\tt tomas.falcato.costa@ist.utl.pt} \\
        Instituto Superior T\'{e}cnico}
\date{\today}


\documentclass[a4paper]{IEEEtran}
\usepackage[utf8]{inputenc}
\usepackage{graphicx}
\usepackage{amsmath}
\usepackage{amsfonts}
\usepackage{listings}
\usepackage{algorithmic}
\usepackage{subfig}
\usepackage[export]{adjustbox}

\begin{document}
\maketitle

\begin{abstract}
 Para o problema de \textit{Financial portfolio optimization}, dado um conjunto de \textit{assets}, pretende-se obter uma forma de distribuir o dinheiro a investir por forma a maximizar o retorno. A cada \textit{asset}, existe um risco associado, sendo que o risco de cada um poderá, ou não, influenciar o risco dos restantes.
%----------------------------------------------------------------------------------------------------------------
%PERGUNTAR!!!!!!!!!!!!!!!!!!!!!!!!!!!!!!!!!!!
Se considerarmos todas as variáveis inerentes à otimização deste tipo de problemas, iremos obter funções não convexas, pelo que se considera uma aproximação facilmente computável através de otimização convexa e cujos resultados, obtidos a partir do \textit{software CVX}, são bastante próximos dos ótimos. 
%----------------------------------------------------------------------------------------------------------------
%segunda parte do projeto
Posteriormente, o problema foi reformulado por forma a obter uma solução sub-ótima mas com menor tempo de computação. 
%----------------------------------------------------------------------------------------------------------------
%principais resultados
%PERGUNTAR!!!!!!!!!!!!!!!!!!!!!!!!!!!!!!!!!!!!!!!!!!!!!
%----------------------------------------------------------------------------------------------------------------
\end{abstract}

\smallskip
\noindent \textbf{Keywords.} Portfolio Optimization \textbullet CVX \textbullet Barrier Method \textbullet Newton Method

\section{Introdução}
\label{sec:introduction}
Este artigo pretende obter uma solução para o problema de \textit{Portfolio Optimization}. Para tal, a nossa abordagem maximiza o retorno esperado e minimiza o risco de investimento tendo em conta a aversão ao risco colocada pela pessoa que investe.

Na vida real, o problema discutido neste artigo tem extrema importância na medida em que praticamente todas as pessoas fazem investimentos do seu dinheiro. A formulação terá em conta não só os casos em que a pessoa pretende rápidos retornos e pretende que estes sejam bastante elevados, independentemente de o risco a ele associado ser relativamente elevado, como também casos cujo objetivo seja obter um resultado de longo prazo em que a aversão ao risco é um fator preponderante face à quantia recebida.

%contributions!!!!!!!!!!!!
Por forma a resolver, e melhor compreender o problema, numa primeira abordagem foi utilizado o \textit{software CVX}. Posteriormente, implementa-se a mesma resolução do problema, mas com este reescrito por forma a utilizar o algoritmo \textit{Barrier Method}. 


O artigo está organizado da forma enunciada de seguida. No capítulo
~\ref{sec:problem-formulation}, é introduzida a função a otimizar e as respetivas variáveis,  capítulo~\ref{sec:cvx} descreve a abordagem para a utilização do \textit{software CVX}, sendo esta alterada para a utilização do \textit{Barrier Method}, descrito no capítulo~\ref{sec:barrier-method}. Os resultados obtidos nas duas fases são discutidos e comparados em~\ref{sec:numerical-results}. As conclusões são apresentadas posteriormente em ~\ref{sec:conclusion}.

\section{Formulação do Problema}
\label{sec:problem-formulation}

Para a obtenção dos resultados do problema descrito anteriormente, a função utilizada foi a seguinte:
\begin{equation}
  \label{eq:problem}
\begin{array}[t]{ll} \text{maximize} & \mu^\top \omega - \gamma \omega^\top \Sigma \omega \\
\text{subject to} & 1^\top \omega = 1 \\ &  \omega \in \mathbb{R}_+^n \end{array}
\end{equation}

Apresenta-se de seguida a descrição das variáveis da função de otimização:

\begin{itemize}
\item  $\omega$ $\rightarrow$ vetor de dimensão \textit{n}, onde cada posição \textit{i}, contém a fração do dinheiro a investir no \textit{asset i} para obter o máximo de lucro. Será esta a variável que se pretende otimizar;

\item $\mu$ $\rightarrow$ vetor de dimensão \textit{n}, onde cada posição \textit{i}, contém o retorno esperado para o \textit{asset i};

\item $\Sigma$ $\rightarrow$ matriz das covariâncias, de dimensão \textit{n}$\times$\textit{n} dos retornos dos \textit{assets}. É através deste parâmetro que se representa a influência que cada \textit{asset} tem nos restantes e vice-versa. Por exemplo, é de esperar que tendo um valor muito elevado na entrada \textit{ij} desta matriz, no caso de haver uma quebra nos retornos do \textit{asset i}, o \textit{asset j} irá, também sofrer quebra nos retornos, ou um aumento dos mesmos, dependendo do valor dessa entrada. Um caso real desta formulação é a forte dependência entre as ações do petróleo e dos automóveis ou a relação de competição entre cadeias concorrentes;

\item $\mu^\top \omega \rightarrow$ representa o retorno expectável do portfólio;

\item $\gamma \rightarrow$ variável que controla o \textit{tradeoff} entre o retorno e o risco. Será este parâmetro que irá controlar se se pretende investir dando preferência a maximizar o retorno, ou dando preferência à minimização do risco;

\item $\gamma \omega^\top \Sigma \omega \rightarrow$ representa a variância do retorno do portfólio. É o parâmetro que mede o risco de investimento;
\end{itemize}

A \textit{constraint} $1^\top \omega = 1$ indica que a soma dos investimentos no portfólio deve corresponder ao capital inicial que se pretende investir. Note-se que os investimentos são uma fração do capital inicial e não o seu valor absoluto. Ao forçar que $\omega \in \mathbb{R}_+^n$ indica que não podem existir investimentos negativos. 
%Falar com o professor
Na prática esta situação poderia existir, no caso de termos capital num dado \textit{asset} e a melhor solução ser retirar o capital deste para investir noutro mais lucrativo.

Ao analisar a função que queremos otimizar, verifica-se que é quadrática, ou seja, do tipo $x^\top Ax+b^\top x+c$.
Para o contexto em questão, tem-se que maximize $\mu^\top \omega - \gamma \omega^\top \Sigma \omega \equiv$ minimize $\gamma \omega^\top \Sigma \omega + (- \mu^\top) \omega \equiv$ minimize $\omega^\top \gamma \Sigma \omega + (- \mu^\top) \omega$.

Sendo a matriz das covariâncias semi-definida positiva e simétrica, verifica-se que a nossa função é realmente quadrática sendo $A = \gamma \Sigma$, $b = -\mu$ e $c=0$. Sendo a função quadrática, podemos também concluir, que é convexa. Desta forma, a otimização será feita de acordo com esta propriedade.

\section{CVX}
\label{sec:cvx}
Por forma a implementar a solução apresentada na equação 1, e sendo esta convexa, utilizou-se o \textit{software CVX}\cite{CVX}. De seguida apresenta-se a parte do código de \textit{Matlab} onde se utiliza o \textit{CVX}.

\lstinputlisting{cvx.m}

\section{Barrier Method}
\label{sec:barrier-method}

Por forma a melhorar o tempo de otimização, utilizou-se um \textit{Interior Point Method}, visto que o problema pertence à categoria dos \textit{Inequality and equality-constrained smooth problems}, categoria esta que é reduzida para uma sequência de \textit{Equality-constrained smooth problems}, a partir da utilização de \textit{Interior Point Methods}. Iremos descrever como fizémos esta transformação ao longo do capítulo.

Para esta abordagem, pretende-se reescrever o problema sob a forma de problema de otimização da seguinte forma:

\begin{equation}
\label{eq.ineq}
\begin{array}[t]{ll} \text{minimize} & f_0(x) \\
\text{subject to} & f_i(x) \leq 0 \\ &  Ax=b \end{array}
\end{equation}

As \textit{constraints} de otimização devem portanto ser lineares ou inequações do tipo acima descrito em que os $f_i$ são convexos e duas vezes diferenciáveis.

Por forma a traduzir a formulação de (\ref{eq.ineq}) numa formulação de um \textit{Equality-constrained smooth problem}, e utilizando \textit{Unconstrained Minimization Technique}, obtem-se (\ref{eq.log-bar}), que é uma aproximação de (\ref{eq.barrier}).

\begin{equation}
\label{eq.barrier}
\begin{array}[t]{ll} \text{minimize} & f_0(x) + \sum_{i=1}^{m} I\_(f_i(x)) \\
\text{subject to} &  Ax=b \end{array}
\end{equation}

\begin{equation}
\label{eq.log-bar}
\begin{array}[t]{ll} \text{minimize} & f_0(x) + (1/t) \sum_{i=1}^{m} log(f_i(x)) \\
\text{subject to} &  Ax=b \end{array}
\end{equation}

Em (\ref{eq.barrier}), $I\_f_i(x)$ é a função indicadora e representa uma barreira para a função de custo, ou seja, para $f_i(x) \leq 0$, tem valor nulo sendo infinito para os restantes. Isto traduz-se na penalização infinita caso as \textit{constraints} se aproximem da violação das condições impostas pelas inequações. Na equação (\ref{eq.log-bar}), $(1/t) log(f_i(x))$ representa a \textit{log barrier}, ou seja, a aproximação da barreira à função indicadora. À medida que \textit{t} aumenta, a \textit{log barrier} aproxima melhor a função indicadora, obtendo-se assim resultados cada vez mais semelhantes à solução ótima. Está ilustrado na figura \ref{fig:barrier} o aumento de \textit{t} e a sua consequência na \textit{log barrier}.

%PERGUNTAR!!!!!!!!!!!!!!!!!!!!!!!!!!!!!!!!!!!!!!!!!!!!!!!!!! CENTRAR
\begin{figure}[htp]
  \centering
  \captionsetup{font=scriptsize} 
  \includegraphics[scale=0.8]{./log-barrier}
  \caption{Aproximação da função indicadora por vários valores de \textit{t}}
  \label{fig:barrier}
\end{figure}

Não sendo possível a implementação de funções não diferenciáveis, faz-se uma aproximação logarítmica, representada em (\ref{eq.log-bar}), da barreira onde é possível aplicar o método de \textit{Newton}. Desta forma, escolhendo uma tolerância e um valor para \textit{t}, será possível obter uma solução sub-ótima para o problema em apenas uma iteração. Utiliza-se o \textit{Barrier Method} que permite efetuar várias otimizações de $f(z)$ em função de \textit{t}.
Quando \textit{t} tende para infinito espera-se que $x^\ast(t)$ tenda para $x^\ast$, e que este valor seja a solução do nosso problema. As soluções de cada iteração estão situadas ao longo do \textit{central path} que corresponde ao conjunto de soluções $x^\ast(t)$ e são caracterizadas pelas \textit{KKT Conditions}. A cada iteração do \textit{Barrier Method}, \textit{outer iteration}, é computada uma solução sub ótima, que é depois utilizada como ponto inicial da próxima iteração. Estende-se esta computação até que o \textit{stopping criterion} seja verificado. 
A cada iteração o valor de \textit{t} é aumentado por um fator $\mu$, que controla o valor que a \textit{log barrier} toma. O \textit{stopping criterion} é obtido através de $f(x^\ast(t))-f^\ast \leq m/t$, em que m é o número de \textit{contraints} existentes no problema.

No \textit{Barrier Method} a escolha dos parâmetros \textit{t} e $\mu$ é de grande importância. Caso $\mu$ seja muito pequeno irão ser necessárias muitas \textit{outer iterations}, visto que o aumento de \textit{t} será reduzido, por outro lado, caso $\mu$ seja muito elevado, irão ser necessárias muitas iterações do método de \textit{Newton}, já que as soluções sub ótimas de iterações consecutivas irão estar muito afastadas entre si. Quanto ao valor de \textit{t} inicial, se este for muito pequeno, irão ser também necessárias muitas \textit{outer iterations}, a adaptação à função indicadora será feita lentamente, caso \textit{t} inicial seja muito elevado, a primeira iteração do método de \textit{Newton} (\textit{first centering step}) irá requerir muitas iterações para calcular $x^\ast$ inicial.
Por último, o \textit{Barrier Method} converge num número finito de iterações.


Pela impossibilidade de aplicação deste método com a formulação anterior devido à \textit{constraint} \: $1^\top\omega=1$, foi necessário recorrer a uma mudança de variável.

Deste modo, define-se $\omega$ pela equação (\ref{eq.w}). 

\begin{equation}
\label{eq.w}
\omega = D z + b
\end{equation}

As matrizes serão as seguintes:
\vskip 4mm
$D_{n,n-1} = 
 \begin{pmatrix}
  1 & 0 & 0 & \cdots & 0 \\
  0 & 1 & 0 & \cdots & 0 \\
  0 & 0 & 1 & \cdots & 0 \\
  \vdots & \vdots & \vdots & \ddots & \vdots \\
  0 & 0 & 0 & \cdots & 1 \\
  -1 & -1 &-1 & \cdots & -1
 \end{pmatrix}
$
\vskip 4mm
$z_{n-1,1} = 
\begin{pmatrix}
  z_{1} \\
  z_{2}\\
  \vdots  \\
  z_{n-1}  
 \end{pmatrix}
$ \: \: \: \: \: \: \:
$
b_{n,1} = 
\begin{pmatrix}
  0 \\
  0\\
  \vdots  \\
  1  
 \end{pmatrix}
$
\vskip 4mm
Com esta mudança de variável, a nova variável de otimização \textit{z}, tem menos um grau de liberdade que a anterior, $\omega$. O último investimento será assim a percentagem em falta do capital inicial. 

Esta condição é conseguida através da equação matricial (\ref{eq.w}), onde a matriz \textit{D} é uma matriz identidade nas primeiras \textit{n-1} linhas e a última linha desta matriz conter apenas "-1"'s; a matriz \textit{b} é uma matriz coluna de "0"'s com  apenas a ultima posição "1", sendo a \textit{constraint} referida anteriormente assim respeitada.

%rever portugues----------------------------------------------------------------------------------------------
Relativamente à segunda \textit{constraint}, $\omega \in \mathbb{R}_+^n$, com a mudança de variável efetuada, tem-se $D z + b \geq 0$. Como tal, o problema descrito pela equação (\ref{eq.ineq}) adaptado à situação em causa ficará escrito da seguinte forma:
%-------------------------------------------------------------------------------------------------------
\begin{equation}
\label{eq.subs}
\begin{array}[t]{ll} \text{minimize} & -\mu^\top (D z + b) + \gamma (D z + b)^\top \Sigma (D z + b) \\
\text{subject to} & -D z - b \leq 0 \end{array}
\end{equation}

Os $f_i(x)$ são assim $f_i(x) = -\delta_i^\top z - b_i$ onde $\delta_i^\top$ representa a linha \textit{i} da matriz \textit{D}.



%AQUI!!!!!!!!!!!!!!!!!!!!!!!!!!!!!

Neste algoritmo, faz-se num primeiro passo a minimização com recurso ao método de \textit{Newton}, sendo a função utilizada a apresentada na equação (\ref{eq.Newton}),

\begin{equation}
\label{eq.Newton}
f(z) = t f_0 (z) + \phi (z)
\end{equation}

onde

\begin{equation}
f_0 (z) = -\mu^\top (D z + b) + \gamma (D z + b)^\top \Sigma (D z + b)
\end{equation}

e

\begin{equation}
\phi = -\sum_{i=1}^{m} log(-f_i(x))
\end{equation}

O termo $\phi$ evita que a minimização com o método de Newton vá para soluções não permitidas, uma vez que a função de otimização assume o valor infinito (\textit{penalty}), obrigando o método de \textit{Newton} a calcular outro \textit{step} para encontrar uma solução viável.

Para a aplicação do método de \textit{Newton}, é ainda necessário o cálculo do gradiente e da hessiana da função $f (z)$. Como tal, o gradiente é obtido da seguinte forma:

\begin{equation}
\nabla f(z) = t (\nabla f_0(z))^\top + \nabla \phi (z)
\end{equation}

onde

\begin{equation}
\nabla f_0 (z) = -\mu^\top D + \gamma (D z + b)^\top 2 \Sigma D  
\end{equation}

e

\begin{equation}
\nabla \phi (z) = \sum_{i=1}^{m} \frac{1}{-f_i (z)} \nabla f_i (z) 
\end{equation}

A expressão da hessiana será obtida de forma idêntica pelas equações (\ref{eq.hess}), (\ref{eq.hesss}) e (\ref{eq.hessss}).

\begin{equation}
\label{eq.hess}
\nabla^2 f(z) = t (\nabla^2 f_0(z)) + \nabla^2 \phi (z)
\end{equation}

\begin{equation}
\label{eq.hesss}
\nabla^2 f_0 (z) = \gamma 2 (D^\top \Sigma D) 
\end{equation}

\begin{equation}
\label{eq.hessss}
\nabla^2 \phi (z) = \sum_{i=1}^{m} \frac{1}{-f_i^2 (z)} \nabla f_i (z) (\nabla f_i (z))^\top
\end{equation}

Com esta reformulação é possível aplicar o \textit{Barrier Method} ao problema de \textit{Portfolio Optimization}.

\section{Resultados Numéricos}
\label{sec:numerical-results}

Nas figuras que irão ser expostas tem-se a vermelho os resultados do \textit{Barrier Method} e a verde do \textit{CVX}.  Os retornos e os riscos dos \textit{assets} são iguais para ambos os programas.
Para verificar o correto funcionamento de ambos os programas realizaram-se testes em que os resultados podem ser facilmente previstos, tais como:

\begin{itemize}
\item Igualar $\gamma$ a zero, ou seja, caso o utilizador não tenha quaisquer preocupações com o risco associado a cada investimento. Espera-se, portanto, que o portfólio obtido seja investir todo o capital no \textit{asset} com maior retorno, independentemente do risco.

Na figura \ref{fig:caso1} observa-se que, de facto, o capital foi investido no \textit{asset} de maior retorno. Existem pequenas diferenças entre o portfólio apresentado pelos dois programas, pelo que se pode observar que no resultado do \textit{CVX} o investimento é total no \textit{asset} de maior retorno, enquanto que no resultado do \textit{Barrier Method} existem pequenas quantias investidas pelos outros \textit{assets}.

\item Igualar $\gamma$ a 100 (100 é um valor arbitrário), pretende-se ilustrar uma aversão ao risco muito elevada por parte do utilizador. Espera-se um portfólio bastante conservador, ou seja, que seja investido capital apenas no(s) \textit{asset(s)} com menor risco associado. Quando $\gamma$ tende para infinito, o portfólio será investido somente nos \textit{assets} com menor risco associado, ao contrário do caso em que $\gamma$ é igual a 100, onde existem pequenos investimentos nos outros \textit{assets}.

\begin{figure}[!htb]
    \centering
    \begin{minipage}{.5\columnwidth}
       \captionsetup{font=scriptsize}  
 		\centering
  		\includegraphics[scale=0.3]{./gama_0}
  		\caption{Portfólio com $\gamma$ igual a 0}
  		\label{fig:caso1}
    \end{minipage}%
    \begin{minipage}{0.5\columnwidth}
       \captionsetup{font=scriptsize}  
  		\centering
  		\includegraphics[scale=0.3]{./gama_100}
  		\caption{Portfólio com $\gamma$ igual a 100 e alguns retornos negativos}
  		\label{fig:caso2.1}
    \end{minipage}
\end{figure}

Na figura \ref{fig:caso2.1} a solução dada por ambos os programas é em tudo semelhante, note-se que, apesar das \textit{rates of return} dos \textit{assets} de menor risco serem negativas, a aversão ao risco do utilizador irá gerar um portfólio em que a importância dada ao risco é bastante superior à dos retornos. A divisão do portfólio pelos dois \textit{assets} de menor risco é a esperada, ambos os programas \: escolheram investir mais capital no \textit{asset} com maior retorno.

\vskip 4mm
\item Igualar apenas os \textit{returns} de todos os \textit{assets}. O portfólio deverá investir a grande parte do seu capital no \textit{asset} de menor risco. Quando se aumenta $\gamma$ nesta situação é esperado que a quantidade investida no \textit{asset} de menor risco aumente.

\begin{figure}[htp]
\captionsetup{font=scriptsize}  
  \centering
  \includegraphics[scale=0.55]{./miu_igual_risco_diferente}
  \caption{Portfólio com retornos iguais para todos os \textit{assets}}
  \label{fig:caso4}
\end{figure}

No caso enunciado acima, a solução de portfólio ótima é intuitiva, os programas deverão investir todo o capital no \textit{asset} de menor risco, visto que todos retornam o mesmo valor do investimento inicial. É isto que se comprova na figura \ref{fig:caso4}.

\item Correlação entre \textit{assets}. Criou-se uma matriz de covariâncias que apresenta uma grande correlação entre o \textit{asset} um e o \textit{asset} dois. Deste modo pretende-se verificar se o risco de investir num dado \textit{asset} influencia a decisão de investir noutro. Para este caso foi utilizado $\gamma=1$.

\begin{figure}[htp]
\captionsetup{font=scriptsize}  
  \centering
  \includegraphics[width=0.9\columnwidth]{./correlacao}
  \caption{Influência de um \textit{asset} noutro}
  \label{fig:corr}
\end{figure}

Na figura \ref{fig:corr} pode-se constatar que apesar dos dois primeiros \textit{assets} serem os que têm maiores retornos, o portfolio acaba por dividir os investimentos entre os \textit{assets} dois e três. De notar que os riscos dos \textit{assets} dois, três e quatro não são muito diferentes, pelo que o investimento no \textit{asset} dois seria uma boa solução. Este investimento não é ótimo pois há uma forte correlação entre o um e dois, e portanto, grande parte do investimento é feito em três.

\end{itemize} 



No gráfico da figura \ref{fig:tempometodos}, é possível observar o tempo requerido para a otimização de uma solução para o problema, consoante a abordagem utilizada, em função do número de variáveis do problema.

Quando o número de variáveis é pequeno, menor que cem, o \textit{Barrier Method} é cerca de cem vezes mais rápido que o \textit{CVX}. Ao aumentar o seu número, a diferença entre os tempos de otimização diminui, continuando o \textit{Barrier Method} a ser mais rápido (cerca de dez vezes). A partir de cerca de mil variáveis, a diferença entre os tempos volta a ser superior.


\begin{figure}[htp]
\captionsetup{font=scriptsize}  
 \raggedright
  \begin{minipage}{5cm}
 \includegraphics[width=9.8cm, height=6cm, left]{./tempo_metodos}
  \end{minipage}
  \caption{Tempo de resolução do problema pelos dois \textit{solvers}}
    \label{fig:tempometodos}
\end{figure}


\section{Conclusão}
\label{sec:conclusion}

A partir da utilização do \textit{Barrier Method} foi possível aproximar os resultados aos obtidos pelo \textit{CVX} só que com um tempo bastante inferior. Os testes realizados foram feitos com número de incógnitas variável de forma a obter uma gama de resultados com maior abrangência, sendo obtidos resultados até mil e quinhentas variáveis, número a partir do qual os resultados deixam de ser produzidos em tempo útil. Sendo possível obter pelo \textit{Barrier Method} resultados em menor tempo, este será um método de eleição para a vida real, em que milissegundos são fulcrais na perda/ganho de capital próprio.


\begin{thebibliography}{9}

\bibitem{CVX}
  [CVX15] \texttt{http://cvxr.com/cvx}, Junho 2015
  
\bibitem{BARRIER}
  [BV\_CVXSLIDES] \texttt{https://web.stanford.edu/boyd/cvxbook/ bv\_cvxslides.pdf}


\end{thebibliography}

\end{document}
This is never printed 